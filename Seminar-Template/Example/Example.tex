\chapter{Titel meiner Seminar-Arbeit}
\markboth{Titel meiner Seminar-Arbeit}{}
\chaptauthors{Mein Name}

\Kurzfassung{%
Dies ist der Abstrakt. Er pa�t ziemlich genau auf diese Seite, ist zumindest
auf keinen Fall l�nger.
}

\newpage

\minitoc %Das Inhaltsverzeichnis

\newpage

\section{Das ist mein erstes Kapitel}

Je nach Talk Nummer wird ausschlie�lich im jeweiligen \texttt{TalkX} Verzeichnis
gearbeitet. Bitte Dateien, Bilder etc. \textbf{nur} im Verzeichnis \texttt{TalkX}
ablegen und bearbeiten. Die Ausarbeitung wird in die jeweilige
\texttt{Seminar-Arbeit.tex} geschrieben. Bitte die Datei \texttt{Example.tex} als 
Grundlage nehmen. Formatierungen, Seiteneinstellungen oder die Datei \texttt{talk.tex} 
d�rfen nicht ge�ndert werden.

Sollte es sich absolut nicht vermeiden lassen, da� weitere usepackages
eingebunden werden m�ssen o.�. bitte \textbf{niemals} die Datei
\texttt{talk.tex} bearbeiten. Daf�r vorgesehen ist die Datei
\texttt{TalkX/MyHeader.tex}. Bitte nur im Notfall bzw. nach
R�cksprache mit dem Betreuer von dieser M�glichkeit Gebrauch machen,
da \LaTeX \ nicht �ber Namensr�ume verf�gt und sich somit Konflikte zwischen
einzelnen Paketen ergeben k�nnen.

\section{Gliederung der Arbeit}

Die Seminar-Arbeit wird in ein Kapitel (\texttt{$\backslash$chapter}) geschrieben. 
Zur Gliederung der Arbeit werden die Befehle
\texttt{$\backslash$section\{\}},
\texttt{$\backslash$subsection\{\}}, und
\texttt{$\backslash$subsubsection\{\}}
verwendet.

Abs�tze m�ssen generell mit einer leeren Zeile getrennt werden und nicht mit
\texttt{$\backslash$$\backslash$} oder \texttt{$\backslash$newline}. 
Bitte die Befehle \texttt{$\backslash$newpage}, \texttt{$\backslash$clearpage} etc. nicht verwenden.

Aufz�hlungen mit und ohne Nummerierung k�nnen mit den folgeneden Befehlen erstellt werden:
\begin{quote}
  \begin{verbatim}
  \begin{enumerate}
    \item ...
    \item ...
  \end{enumerate}

  \begin{itemize}
    \item ...
    \item ...
  \end{itemize}
  \end{verbatim}
\end{quote}

F�r Beschreibungen kann der folgende Befehl benutzt werden:
\begin{quote}
  \begin{verbatim}
  \begin{description}
    \item[Begriff] Beschreibung
    \item[Begriff] Beschreibung
  \end{description}
  \end{verbatim}
\end{quote}


\section{Bilder und Tabellen}

Bitte \textbf{alle} Bilder ohne Dateinamenendung einbinden und das jeweilige
Bild als \texttt{.jpg} oder \texttt{.pdf} im Verzeichnis \texttt{TalkX} ablegen. 
Zum Einbinden kann der folgende Befehl verwendet werdet:
\begin{quote}
  \begin{verbatim}
  \begin{figure}[ht]
    \begin{center}
    \includegraphics[scale=0.6]{TalkX/filename}
    \end{center}
    \caption{Beschriftung}
    \label{label}
  \end{figure}
  \end{verbatim}
\end{quote}

\begin{figure}[ht]
	\begin{center}
  \includegraphics[scale=0.6]{Example/uzh_logo_d}
  \end{center}
  \caption{Beschriftung}
  \label{fig:label}
\end{figure}

Beim Einbinden den relativen Pfad angeben und keinen absoluten!
Anstatt \texttt{[scale=0.6]} k�nnen auch die Parameter \texttt{[width=4cm]} oder
\texttt{[width=0.6$\backslash$textwidth]} f�r die Skalierung der Bilder verwendet werden.
In der schriftlichen Ausarbeitung m�ssen die Bilder mit einer Mindestaufl�sung
von 600dpi erstellt werden.

  \begin{table}[h]
    \caption{Beschriftung}
    \label{tab:label}
  	\begin{center}
    \begin{tabular}{|c|c|c|c|} \hline
	          & A & B & C \\ \hline\hline
	        X & 1 & 2 & 3 \\ \hline
	        Y & 4 & 5 & 6 \\ \hline
	        Z & 7 & 8 & 9 \\ \hline
	  \end{tabular}
	  \end{center}
  \end{table}

Die Tabelle \ref{tab:label} kann z.B. mit dem folgenden Befehl erstellt werden:
\begin{quote}
  \begin{verbatim}
  \begin{table}
    \caption{Beschriftung}
    \label{tab:label}
    \begin{center}
    \begin{tabular}{|c|c|c|c|} \hline
	          & A & B & C \\ \hline\hline
	        X & 1 & 2 & 3 \\ \hline
	        Y & 4 & 5 & 6 \\ \hline
	        Z & 7 & 8 & 9 \\ \hline
    \end{tabular}
    \end{center}
  \end{table}
  \end{verbatim}
\end{quote}

Die Bilder und Tabellen m�ssen grunds�tzlich mit einer Beschriftung (\verb|\caption|)
versehen werden, und im Text mit \texttt{$\backslash$ref\{label\}} referenziert werden.
Das \texttt{caption} ist bei Bildern grunds�tzlich unten und bei Tabellen oben!


\section{Literaturverzeichnis}

Am Ende des Kapitels wird das Literaturverzeichnis erstellt. Bei den Bibitems
wird \textbf{keine} Marke angegeben, es werden die automatisch erzeugten
Marken [1],[2],... verwendet. Bei einer Referenz m�ssen grunds�tzlich die
Autoren, der Titel, der Verlag und das Erscheinungsdatum in dem folgenden Format 
angegeben werden:
\begin{verbatim}
\bibitem {label} N. Author: Title of the document; Type of document 
  (technical report, deliverable, Workshop/Conference Name ...), 
  (Location, Vol. X, No. Y), Month, Year, pages, URL (if available).

\bibitem {label} Website title; \url{Website URL}, Month, Year of last visit.
\end{verbatim}
Wenn in der Referenz eine Internetadresse benutzt wird, muss diese mit dem
Befehl \texttt{$\backslash$url\{http://...\}} angegeben werden.

Im Text werden die Bibitems mit \texttt{$\backslash$cite\{label\}} referenziert.
F�r alle verwendeten Arbeiten und Bilder m�ssen an der jeweiligen Stelle Referenzen
eingef�gt werden.

Eine ausf�hrliche Anleitung zur Referenzierung in wissenschaftlichen Arbeiten 
findet sich im \emph{Leitfaden zur Erstellung schriftliche Seminararbeiten} \cite{leitfaden}. 

\section{Kompilieren}

\LaTeX \ ist bei allen g�ngigen Linux Distributionen dabei. Unter Linux wird 
das Dokument mit \texttt{pdflatex talk.tex} im Hauptverzeichnis kompiliert, 
was die Datei \texttt{talk.pdf} erzeugt. 

F�r Windows ist die \TeX \ Implementation, MiKTeX (\url{http://www.miktex.org/}) zusammen mit dem \LaTeX \ Tool, TeXnicCenter (\url{http://www.toolscenter.org/}) zu empfehlen.

Probleme, Anregungen und Fragen bzgl. der Anfertigung des Dokumentes richten
Sie bitte per Email an die Betreuer.

Zum Abgabetermin ist es dann nur notwendig das Verzeichnis
\texttt{TalkX} zu packen (mit \texttt{zip} oder \texttt{tar}) und per
Email an die Betreuer zu senden.


\begin{thebibliography}{99}
\bibitem {leitfaden} Martin Waldburger, David Hausheer, Burkhard Stiller: \emph{Leitfaden zur Erstellung schriftliche Seminararbeiten}, Communication Systems Group, IFI, University of Zurich, Februar 2007. \url{http://www.csg.uzh.ch/teaching/hs11/inteco/extern/Leitfaden_Seminararbeiten.pdf}
\end{thebibliography}

