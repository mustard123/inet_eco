%!TEX root = ../talk.tex

\chapter{Comparison of IoT Business Model}
\markboth{Comparison of IoT Business Model}{}
\chaptauthors{Matthias Diez, Christian Ott, Silas Weber}

\Kurzfassung{
	\begin{itemize}
		\item Explain why IoT and business models for it are relevant
		\item Present the method that is used to gain the results
		\item Summarize the results of the paper
	\end{itemize}
}

\newpage

\minitoc %Das Inhaltsverzeichnis

\newpage
\renewcommand{\labelitemii}{$\diamond$}
\renewcommand{\labelitemiii}{$\circ$}
\section{Introduction and Problem Statement}
The goal of this paper is fourfold, first a common ground is built, with the definitions and characterizations of IoT, business models and business model frameworks. Second, the basic business model framework concepts ``magic triangle'' and  Business Model Canvas are introduced. Third, the changes that were made to business model frameworks due to the development if IoT are highlighted, and lastly, the currently available IoT business model frameworks are presented and compared against each other.\\
The rest of the paper is structured as follows. In Section 1.2 the term `Internet of Things' is introduced and its definition, growth, architecture and potential of IoT for business evaluated. Section 1.3 presents the concept of business models and business model frameworks. Along with the definitions, two major `classic' business model frameworks are described. Section 1.4 shows available business model frameworks that are adapted for IoT, Section 1.5 will compare those frameworks and illustrate the results based on a model company. Finally, in Section 1.6 will evaluate the findings of the previous sections, the summery and conclusion will be presented in Section 1.7. 
 
\section{Internet of Things}
The term ``Internet of Things'' (IoT) has grown to one of the most discussed topics in academia and industry \cite{ju}. The phrasing was first used by Kevin Ashton during a presentation in 1999. From there IoT has become a new paradigm, it basically means the ``interconnection of physical objects, by equipping them with sensors, actuators and a means to connect to the Internet'' \cite{dijkman}.

Even though the term ``Internet of Things'' is currently used by everyone, there is no common consent about its definition. The International Telecommunication Union (ITU) defined IoT 2012 in a recommendation as ``a global infrastructure for the information Society, enabling advanced services by interconnecting (physical and virtual) things based on, existing and evolving, inter-operable information and communication technologies'' \cite{itu}. Besides their view on IoT, other definitions were built, Atzori et al. identified three visions how IoT may be seen. As illustrated in Figure 1.x one is focused on the `things' getting connected over technologies like RFID, WLAN or NCF. The second visions is `Internet oriented', where anything connects with anyone . The third point of view is a `semantic oriented' vision, definitions in this direction can be found in literature. The the semantic oriented vision present thoughts on problems generated through the extremely increased number of connected `things', e.g. the challenging issues related to the representation, storage, interconnection, search and organization of information from IoT \cite{atzori}.\\
Due to the differences in these visions, the definitions of IoT established by organizations or entities vary a lot depending on their specific interests, approach taken on the subject as well as their backgrounds. The convergence of these three visions illustrated in Figure 1.x can be seen as the overall paradigm of IoT, this is also the perspective this paper takes  on in the remaining sections. 
  \begin{figure}[h]
    \begin{center}
    \includegraphics[scale=0.35]{Talk11/iot_visions.jpg}
    \end{center}
    \caption{``Internet of Things'' paradigm as a result of the convergence of different visions.}
    \label{``Internet of Things'' visions}
  \end{figure}

The number of worldwide connected `things' is rapidly growing, in 2016 Gartner predicted up to 6.4 billion connected devices which will reach up to 20.8 billion by 2020 \cite{gartner}. Consequently organizations are more and more involved in IoT-related products, applications and services either in the development or as an investment \cite{ju}. Google acquired Nest for \$3.2 billion, a quickly growing smart thermostat company. Samsung bought SmartTings, an open smart home platform []. A lot of IoT companies emerge like LiFX as smart bulb company [] or TODO. Telecommunication organizations invest in future technologies like 5G which was discussed in chapter ?? of this seminar report. Governments increasingly acknowledge the importance of IoT, the United States supported a Smart Cities Initiative with \$160 million [], the Korean government planned \$5 billion investments in various IoT projects until 2020 []. The investments driven by national institutions as well as aggressive investments by companies in IoT would ``create new business opportunities and substantial social and economic benefits''[]. 

Even though there is no concrete definition what IoT is, there is some consent about the architecture of IoT. A Three layered application stack can be found in various research literature albeit the layers may be named differently they are semantically comparable [fleisch, ju, wortmann, ]. A `thing' layer, which means a hardware-based sensing layer, it has the function to identify objects and collect information through sensors over short-rang and local networks \cite{ju}.\\
The network layer transmits real-time data. It interconnects not only people to `things' but allows also an information flow between `things' autonomously. The data presented by the network layer can be used by companies to provide optimized and personalized services \cite{ju}\\
The application layer is described by Ju et al. as a ``combination of data processing and intelligence analysis to meet the industry needs to realize an intellectualized industry''. The application layer allows companies to achieve ``different types of intelligent application solutions'' as well as the determination of business strategies\cite{ju}.

The Internet of Things has a huge potential. On one side are the customers which enjoy new experiences through the new connected world, on the other side are the businesses getting involved in the IoT market, they hope \cite{ju}.

\section{Business Model Frameworks}
	
	\subsection{Definitions}
	 A \textbf{Business Model} is ``a description of the value a company offers to one or several segments of customers and of the architecture of the firm and its network of partners for creating, marketing and delivering this value and relationship capital, to generate profitable and sustainable revenue streams'' [osterwalder 2005]. The concept of the business model became important in the 1990 as the Internet began to spread and became important since then [22]. There is no commonly accepted view what the business models should include [morris, osterwalder, schweizer]. Achtenhagen et al. stated that there has been a change from ``what business models are'' to ``what business models are for'' \cite{westerlund}. In various literature there seems to exist an agreement that a business model is `the way of doing business' for a particular firm \cite{westerlund}. The heart of each business model stands the goal of minimizing cost and maximizing revenue \cite{ju}.\\
	 Business models are composed of a variety of components. In business models commonly found components are customer segments, value proposition, channels, customer relationships, revenue streams, key resources, key activities, key partnership and cost structure[ osterwalder 2005]. These Components are the fundamental building blocks of the Business Model Canvas and, thus they are described in detail in Section 1.xyz. \\

	 A \textbf{Business Model Framework} is a tool to help a company in the development of its business model by presenting an overview over the business model components described above \cite{dijkman}.
	\begin{itemize}
		\item Definitions
		\item ``Revolutionizing the Business Model''
			\cite{gassmann}
			\begin{itemize}
				\item Why design Business Models?
				A successful business needs to offer products or services that there is a demand for and thus can be sold to make a profit. In short, a business has to create value. A company needs to have a strategy, a way of conducting business on an operational level that will result in long term financial success, else a business is not viable and will disappear from the market sooner or later.
				A business model can be seen as a logical abstraction describing how the business operates to make a profit. A business models hence is a core factor in deciding whether your business will be driven out of market or prosper. Having a suitable business model is essential. In fact, innovators of business have been found to be 6\% more profitable than product or process innovators (BCG 2008). With the spread of IOT new business models need to be developed to adapt to those technological changes in the environment and seize the opportunities that arise witch such a change. 

				But there seems to be a problem. Few managers can explain their company's business model ad-hoc, let alone define what a business model actually is (Gassmann et al.)[1]. They employ a simple conceptualization consisting of four central dimensions: Who, What, How, Value. 
				\item Magic Triangle 

				\begin{itemize}
					\item What is the Magic Triangle?

					The Who, What, How and Value make up the Gassmann's magic Triangle. 
					The Who addresses the target customer Group. 
					The What refers to the product or service to those customers. What is the value offered to the customer? 
					The Value explains how the business model is financially viable. How is the value generated? It includes cost and revenue structures.
					The How addresses the process and activities as well as the resources and capabilities that are required for building and distributing the value propositions.
					
					
						\begin{figure}[ht]
						    \begin{center}
						    \includegraphics[scale=0.6]{Talk11/Figure1.png}
						    \end{center}
						    \caption{the magic triangle}

						\end{figure}
						    \label{label}
					Figure 1.1 shows the constellation of the magic triangle with the ``Who'' node in the middle.
					Gassman argues that reducing the business model to those four components allows for a simple, yet thorough enough view of the business model architecture.



				\end{itemize}
				\item Creating new Business Model

				Creating new business models is not an easy task. It requires thinking outside the box, going beyond convention industry philosophy and can quickly become complex. People that are versed in change management will recognize some of the problem. First, why even come up with a new or improved business model? As long as there are still profits coming in, why take a risk and leave your comfort zone? Competition and environmental factor are always in movement. What works today may not work so well anymore in a few years. The syndrome known as the boiling frog syndrome states that gradually increasing problems are go unnoticed and or not dealt with until until it's too late to tackle them. A frog in pot with water won't jump out if the water is slowly brought to boil and when the water gets too hot for him, his muscles are too weak to jump out. Likewise, a business that has diminishing returns year after year may not be concerned about the the little loss between consecutive year. But when they realize the devastating loss suffered over multiple years there aren't any resources left to tackle the problem and they go bankrupt.

				Another problem is the not invented here syndrome, meaning that ideas not coming from within the company are disregarded solely because they come from the outside. As a consequence, business models should not just be copied from somewhere else but rather bring in external stimuli when generating ideas from within.

				Gassman analyzed 250 business models in different industries from the last 25 years and as a result identified 55 pattern of business models that have served as the basis for new business models in the past. Then, together with selected companies they developed a construction methodology based on the fact that 90\% of all new business models have recombined previousley existing ideas, concepts and  technologies.
				The BMI Navigator is the ready-to-use methodology for coming up with new business models consisting of the following three steps:


				\begin{itemize}
				 	\item Initiation
					Describing the current business model is a good starting point. It creates a common ground for discussing the things that are done well and what needs improvement, and opportunities are open to be exploited. Also it get the participants started thinking in the ways of business models.
					Open-minded team members are essential, preferably from different functions. This allows different viewpoint and thinking outside the box as well as overcoming the prevailing industry logic.
					Recombining existing ideas helps generating new business models. Gassman condensed for this puropose the 55 business models into a set of pattern cards as shown in figure 1.2. Each card has a title, a description and an example. The goal is applying different cards to the current model to see what would change in this situation. The cards should trigger discussions and act as a stimuli for new innovative ideas.

				 \item Ideation


					\begin{figure}[ht]
						    \begin{center}
						    \includegraphics[scale=0.6]{Talk11/Figure2.png}
						    \end{center}
						    \caption{pattern card}
						    \label{label}
						\end{figure}



					 \item Integration

					The last step is the integration. It's obvious that a newly genearated idea cannot be implemented instantaneously. New ideas need to be gradually fleshed out into fully operational business models. Considering the new stakeholders, partners and consequences for the market is crucial.


				\end{itemize}
			\end{itemize}
		\end{itemize}	
	\subsection{Business Model Canvas} 
		The Business Model Canvas is a strategic management tool that was developed by Alexander Osterwalder in his doctoral thesis in 2004 and later published in book form by \cite{osterwalder}. The tool supports business developers in gaining an overview about the key factors of their business. There are nine key factors explained by Osterwalder in detail and they are called `building blocks'. All of the building block can be aligned next to each other in a way that visualizes the mechanics of the business models (i.e. the interactions between the building block) and makes it easier to understand and discuss. The analog, pen and paper nature of the business model canvas allows it to be filled out in a team in order to stimulate creativity while finding a suitable business model for the business idea at hand.

		The usual order of filling out the business model canvas \cite{bmc} is by starting with the targeted `Customer Segments'', then following with the `Value Proposition' to these customers and the `Channels' that allow us to deliver the described value to the customer. The value is not only delivered in a physical but there is an emotional `Customer Relationship' with the customer, too that needs to be considered. In exchange for the proposed value, customers are willing to pay money which make up the `Revenue Streams' of the business model. With this fifth building block, the money-generating revenue side of the business model canvas is complete.

		A business model always includes a cost side. In Osterwalder's canvas, the cost side consists of the following four building blocks: The `Key Resources' are human, financial, physical and intellectual resources that are critical to deliver the proposed value. But resources alone do not create value - they need to be used for `Key Activities' in order for the business model to be viable. As a business owner, the decision of what part of the value you create yourself and what you let other companies do for you is written down in the `Key Partners' building block. The costs induced by resources, activities and partners are finally collected in the `Cost Structure' building block.

		A `Customer Segment' can be a mass market, a nice market or a multi-sided platform. Business can have very diversfied customer segments, with clear or unclear segmentation. It can be important to focus on the most important customers first.

		`Value Propositions' can have various characteristics such as the newness, performance, cost reduction potential or accessibility of the proposed value. The key questions to answer with this block is `What value do we deliver to the customer' and `Which customer needs are we satisfying?'. There can be multiple different value propositions per company.

		A company usually offers multiple `Channels' to serve their customers. Different customer segments may require different channels to get the product or service that the company offers. Also, the channel may differ along the different channel phases from Awareness, Evaluation, Purchase, Delivery to After sales.

		`Customer Relationships` differ strongly from company to company. Some forms of business models rely heavily on a personal customer relationship (e.g. barbershop) whereas for other companies the customers might not even be known because they help themselves in a self-service manner.

		`Revenue Streams` can stem from different sources and pricing models. A company can generate income through asset sale, fees for the usage, subscription, brokerage and licensing, or through advertising. Pricing models can be fixed (per feature, customer segment or volume) or dynamic (everyone gets a different price depending on the current context).
 
		`Key Activities' include all actions that are required to offer the `Value Propositions', to run the different `Channels` and maintain the `Customer Relationships' all of which generate the `Revenue Streams`.

		`Key Resources' include the physical, intellectual, human and financial assets of the company. The company pools these resources in order to turn them in an added value that can then be sold to customers.

		The decision to include `Key Partners' into a company's business model can have various reasons. It might be that the partner can provide a needed component of the model that is cheaper, that includes less risk and uncertainty for the company or that can not be provided by the company itself.

		The `Cost Structure' can be characterize along an axis from Cost driven (lowest costs in the market) to Value driven (best product/service in the market). Important characterizations of the cost structure include fixed and variable costs as well as the influence of economies of scale (the more you sell, the lower the cost per item) and economies of scope (the more diverse your offering, the more efficient a company can use its resources).		
\section{Available IoT Business Model Frameworks}
	In this chapter, we will present and discuss different IoT Business Model Frameworks that are currently available in research papers.
	\subsection{IoT Business Model Patterns (Fleisch et. al)}
		Specific business model patterns can be found in the IoT environment were described in \cite{fleisch}. This is a good starting point to find out what areas an IoT business model framework has to cover. The analysis of 55 distinct business model patterns from \cite{gassmann55} yields two novel business model patterns that can be applied to an IoT business: `Digitally Charged Products' and `Sensor as a Service'.

		The pattern of a `Digitally Charged Products' is described as `classic physical products are charged with a bundle
		of new sensor-based digital services and positioned with new value propositions' in \cite[p. 10]{fleisch}. The components that cab be used and combined are `Physical Freemium' (free digital service with a paid product and offering paid premium services), `Digital Add-on' (digital services are sold to customer on an add-on basis to increase the value of the base product), `Digital Lock-in' (limit the digital compatibility), `Product as Point of Sales', `Object Self Service', `Remote Usage and Condition Monitoring'. 

	\subsection{Adapted Business Model Canvas for IoT (Dijkman et. al)}
		Dijkman et al. analyzed current research literature in order to create a business model framework for IoT applications. They searched for papers that contained the phrasing ``Internet of Things'' together with ``business model'', from the resulting 20 papers were 5 papers selected, only the ones that contained actual business models \cite{dijkman}. Two of the five papers were based on the business model canvas described in Section 1.4.

		Dijkman then identified the building blocks and building block types of business models through the selected papers and interviews with professionals working in the IoT business. Lastly he determined the relative importance of each building block or type through a survey.

		The result from Dijkman et al.'s research can be seen in Figure 1.z. The building blocks observed were equal to the building blocks from the business model canvas described in Section 1.4. The building block types differentiate the framework form classic canvas.

		Based on the interviews and a survey the relative importance of each building block was determined. 

		Dijkman's work showed, that `value proposition' is most important in IoT business models. Also as highly important were `customer relationship' and `key  partnerships' considered. 
		\begin{figure}[h]
			\begin{center}
		    \includegraphics[scale=0.52]{Talk11/iot_canvas_rel_imp_dijkman.jpg}
		    \end{center}
		    \caption{Business model framework for IoT applications with relative importance of specific types. \# < 0.05 significance, * < 0.02 significance, **  < 0.01 significance.}
		    \label{Business model for IoT}
		\end{figure}

		\subsection{Value Design (Leminen et al.)}
		Westerlund moves from seeing IoT mainly as a technology platform to viewing it as a business ecosystem. Thus there is a shift from the traditional business model of a firm to designing ecosystem business models. Such an ecosystem business model is composed of value pillars, creating and capturing value. Westerlund identified three major challenges for designing ecosystem business models namely diversity of objects, immaturity of innovation and unstructured ecosystems.
As businesses from centralized toward decentralized and distributed network structures, they become part of a complex business ecosystems. A business ecosystem can be seen an organization of economic actors. Those actors' business activities are anchored around a platform (Muegge 2011). It's is argued that such systems are more than the sum of its part and hence operations of the system cannot be understood by studying its parts detached from the entity (Westerlund).
Existing business model frameworks such as the magic triangle and the business model canvas described above are arguably not adequate enough when it comes to analyzing such ecosystems. With IoT the interdependency of actors in an ecosystem gains importance due to the networks inherent to such ecosystems.
As for all business, making money is essential. Westerlund identified three problems when it comes to making money in the internet of things.
Diversity of objects
Diversity of object refers to the variety of different types of connected objects. Without a widespread standardization, it will be difficult to be efficient in an ecosystem. Managers will face a difficulty when trying to bring the objects, businesses and consumers together. Things can integrate with other things, requiring specific business logics.
Immaturity of innovation
Immaturity of innovation refers to the current multitude of emerging technologies and components as well as innovations that have not yet matured into products and services. For IoT to be successful, modularized objects with of a ``plug and play'' type are needed.
Unstructured ecosystems
With unstructured ecosystems the problem of lacking governance and underlying structures is addressed. Unstructured ecosystems may lack essential participants for example IoT operator or potential customers. New business opportunities arise when new relationships in new industries are built and already existing connections are extended,

Westerlunds ecosystem business model framework helps managers designing feasible business models that overcome these previously discussed problems and fit in the ecosystem nature of IoT. 
The proposed framework consists of four key pillars, value drivers, value nodes and value exchanges.
Value drivers
There are different value drivers in an ecosystem. Those value drivers are composed of both individual and shared motivation of the participants in the ecosystems. The shared motivations i.e. the shared value drivers are crucial for creating a win-win trustworthy ecosystem. Without a long term relationship between the actors with mutual respect of their business goals, the ecosystem will fail. Each value driver also serves as an individual value node's motivational factor. Examples of shared value drivers may be cybersecurity and improved customer experience.
Value nodes
Value nodes consist of various actor, activities and processes linked with other nodes to create value. Further, these nodes also have autonomous actors. Here come the smart things into play. These autonomous actors may be sensors, smart machines or other intelligence. Value nodes are heterogeneous. They could be organizations, networks of organizations or even network of networks.
Value exchanges
Value Exchanges are the exchange of value between but also within different value nodes in the system. The value can be resources, knowledge and information thus they are tangible as well an intangible. Value exchanges are best described as a flow that powers the engine. Value exchanges are important because they describe how revenues are generated and distributed in the ecosystem.
Value extract
As not all created value is meaningful in regard to commercialization, only specific parts of created value make sense to extract. Value extract refers to the part of the ecosystem that extracts value. That means it focuses on what can be monetized and what the respective nodes and exchanges are for the value creating and capture. The concept of value extract is useful to have a restricted view on what is actually relevant in the ecosystem to monetize on. Each business in the ecosystem needs to have something that is beneficial for them from the business point of view. Value extract can be single activities, automated processes, individuals, commercial organizations, non-profits or even groups of organisations or networks with their respective value flows between their nodes.

Westerlunds business model framework is heavily focused on the value part of business models. Under the concept of value design those four value pillar described above come together in a single picture. Value design is an architecture mapping the foundational structure of the ecosystem business model. It provides boundaries for an ecosystem and gives a pattern of operation.


		\subsection{Three-dimensional collaborator Model (Chan)}


	% \begin{itemize}
	% 	\item Explain changes in Business Models due to IoT Development
	% 		\begin{itemize}
	% 				\item From ``Designing Business Models for the Internet of Things'' \cite{westerlund}
	% 			\begin{itemize}
	% 				\item Explains how ecosystem business models can be designed (Value Design) to show ``the dynamics between the components'' or ``ow the engine works''

	% 				\begin{enumerate}[I]
	% 					\item \textbf{Value Drivers}
	% 					\begin{itemize}
	% 						\item Individual motivations
	% 						\item Shared motivations
	% 					\end{itemize}
	% 					\item \textbf{Value Nodes}
	% 					\begin{itemize}
	% 						\item Actors
	% 						\item Activities
	% 						\item (Automated) processes
	% 					\end{itemize}
	% 					\item \textbf{Value Exchanges}
	% 					\begin{itemize}
	% 						\item Resources
	% 						\item Knowledge
	% 						\item Money
	% 						\item Information
	% 					\end{itemize}
	% 					\item \textbf{Value Extracts}
	% 					\begin{itemize}
	% 						\item Monetization
	% 					\end{itemize}					
	% 				\end{enumerate}
	% 			\end{itemize}
	% 			\item From ``IEEE-SA Internet of Things (IoT) Ecosystem Study'' \cite{cisco}
	% 				\begin{itemize}
	% 					\item IoT Ecosystem
	% 						\begin{enumerate}[I]
	% 							\item \textbf{Where does IoT stand today}
	% 								\begin{itemize}
	% 									\item Players
	% 									\item Market
	% 									\item Technology
	% 									\item Standardization
	% 								\end{itemize}
	% 							\item \textbf{Business Model View}
	% 								\begin{itemize}
	% 									\item New segments
	% 									\item Open issues / Insufficiency
	% 								\end{itemize}		
	% 						\end{enumerate}
	% 				\end{itemize}
	% 			\item From ``Business Models and the Internet of Things'' \cite{fleisch}
	% 				\begin{itemize}
	% 					\item Based on 55 different business models patterns \cite{gassmann}
	% 					\item IoT Business Model 1: Digitally Charged Products
	% 						\begin{itemize}
	% 							\item A digitally charged product ``links digital services to physical products to create a hybrid bundle that is a single whole.''
	% 							\item \textbf{Components}
	% 								\begin{enumerate}[a]
	% 									\item Business Model View
	% 									\item Physical Freemium
	% 									\item Digital Add-on
	% 									\item Digital Lock-in
	% 									\item Object Self Service
	% 									\item Remote Usage and Condition Monitoring
	% 								\end{enumerate}
	% 						\end{itemize}
	% 					\item IoT Business Model 2: Sensor as a Service
	% 						\begin{itemize}
	% 							\item A sensor as a service business model is based on ``collecting, processing and selling for a fee the sensor data [...]''.
	% 						\end{itemize}
	% 				\end{itemize}
	% 		\end{itemize}
	% \end{itemize}

\section{Comparison of available IoT Business Model Frameworks using a case study}
	In the following section, we will introduce a fictitious company in order to make it possible to compare the available business model frameworks and illustrate their respective strengths and weaknesses.

	\subsection{Introduction to the case study company}
	Our case study company is called `FlexSpace' that offers a integrated beacon and software solution for companies to make the usage of companies' office space more flexible by showing the available desk places in an office to employees. This allows the customer companies to increase the usage of their office and thereby reduce the square meters per employee which directly saves money for the company. A beacon is a miniature, battery-powered sender that usually uses the Bluetooth Low Energy (BLE) technology to communicate with nearby devices. These beacons allow to let the employees check-in at their desk using their smartphones. `FlexSpace' offer the installation and maintenance of the described beacon infrastructure at customer sites as well as a whitelabel mobile application for their customers. The application displays the available desk places and enables the user to check-in at a specific desk using the closest beacon available. The visual appearance can be customized per customer to fit the customer company's corporate design.

	%\subsection{IoT Business Model Patterns (Fleisch et. al)}

	\subsection{Adapted Business Model Canvas for IoT (Dijkman et. al)}

	\subsection{Value Design (Leminen et al.)}

	\subsection{Three-dimensional collaborator Model (Chan)	}		

	\subsection{Conclusion of the applicability of the three models}
	% \begin{itemize}
	% 	\item Text with explanations of differences between frameworks in Chapter 5
	% 	\item If applicable: Comparison table along different attributes
	% 	\item Introduction into case company (fictitious example)
	% 	\item Present model of case company for each framework
	% \end{itemize}

\section{Evaluations and Discussion}
	% \begin{itemize} 
	% 	\item Partly based on outcome of discussion within the seminar
	% 	\item Critical questions regarding the relevance of the results
	% 	\item How can the conducted comparison help IoT companies to design better business models?
	% \end{itemize}

\section{Summary and Conclusions}
	% \begin{itemize} 
	% 	\item Summarize the paper content
	% 	\item Show limitations of the paper
	% 	\item Give indications for potential future work related to the topic of this paper
	% \end{itemize}

 \begin{thebibliography}{99}
	 \bibitem {gassmann} O. Gassmann, K. Frankenberger, and M. Csik: \emph{Revolutionizing the business model.} in O. Gassmann and F. Schweitzer (Eds.), Management of the Fuzzy Front End of Innovation. Springer, New York, pp. 89-97, 2014, \url{http://link.springer.com/chapter/10.1007%2F978-3-319-01056-4_7}, last visit: October 06, 2016.

	 \bibitem {gassmann55} Gassmann et al.: \emph{Gesch?tsmodelle entwickeln: 55 innovative Konzepte mit dem St. Galler Business Model Navigator.} in Hanser Verlag, 2013.

	 \bibitem {osterwalder} A. Osterwalder, Y. Pigneur, A. Smith: \emph{Business Model Generation}, self published, 2010.

	 \bibitem {bmc} Strategyzer AG: \emph{The Business Model Canvas}, \url{https://strategyzer.com/canvas/business-model-canvas}, last visit: October 06, 2016.

	 \bibitem {dijkman} R.M. Dijkman, B. Sprenkels, T.Peeters, and A. Janssen: \emph{Business Models for the Internet of Things}, International Journal of Information Management, Vol 35, pp 672-678, 2015.

	 \bibitem {fleisch} E. Fleisch, M. Winberger, F. Wortmann: \emph{Business Models and the Internet of Things}, Bosch Internet of Things \& Services Lab, pp 1-19, August 2014, \url{http://www.iot-lab.ch/?page_id=10543}, last visit: October 06, 2016.

	 \bibitem {rossi} B. Rossi: \emph{How the Internet of Things is Changing Business Models}, Information Age - Insight and analysis for IOT leaders, May 4, 2016 \url{http://www.information-age.com/it-management/strategy-and-innovation/123461371/how-internet-things-changes-business-models}, last visit: October 06, 2016.

	 \bibitem {hui} G. Hui: \emph{How the Internet of Things Changes Business Models}, Harvard Business Review, July 24, 2014, \url{https://hbr.org/2014/07/how-the-internet-of-things-changes-business-models}, last visit: October 06, 2016.

	 \bibitem {cisco} CISCO: \emph{IEEE-SA Internet of Things Ecosystem Study}, IEEE Standards Association, New York, 2015, \url{http://www.cisco.com/c/dam/en/us/solutions/collateral/industry-solutions/dlfe-670918525.pdf}, last visit: October 06, 2016.

	\bibitem {westerlund} M. Westerlund, S. Leminen and M. Rajahonka: \emph{Designing Business Models for the Internet of Things} Technology Innovation Management Review, July 2014, \url{http://timreview.ca/article/807}, last visit: October 06, 2016.

	\bibitem {ju} Jaehyeon Ju, Mi-Seon Kim and Jae-Hyeon Ahn: \emph{Prototyping Business Models for IoT Service}, Procedia Computer Science, 91, pp. 882 - 890, 2016, \url{http://www.sciencedirect.com/science/article/pii/S1877050916312911}, last visit: October 06, 2016.
	\bibitem {itu} TODO
	\bibitem {Atzori} TODO
	\bibitem {osterwalder2005} TODO
 \end{thebibliography}

