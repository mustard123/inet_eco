%!TEX root = ../talk.tex

\chapter{Comparison of IoT Business Model}
\markboth{Comparison of IoT Business Model}{}
\chaptauthors{Matthias Diez, Christian Ott, Silas Weber}

\Kurzfassung{
	\begin{itemize}
		\item Explain why IoT and business models for it are relevant
		\item Present the method that is used to gain the results
		\item Summarize the results of the paper
	\end{itemize}
}

\newpage

\minitoc %Das Inhaltsverzeichnis

\newpage
\renewcommand{\labelitemii}{$\diamond$}
\renewcommand{\labelitemiii}{$\circ$}
\section{Introduction and Problem Statement}
	\begin{itemize}
		\item Introduction into the topic
		\item Research Questions / Problem Statement
		\item Structure of Paper
	\end{itemize}

\section{Internet of Things}
	\begin{itemize}
		\item Definition(s) \cite{ju}
		\item Examples of IoT companies
		\item What we understand as IoT in this paper
	\end{itemize}

\section{Business Model Frameworks}
	\begin{itemize}
		\item Definitions \cite{bmc} \cite{dijkman}
			\begin{itemize}
				\item Business 
				\item Business Model 
				\item Business Model Framework
			\end{itemize}
		\item ``Revolutionizing the Business Model'' \cite{gassmann}
			\begin{itemize}
				\item Why design Business Models?
				\item Magic Triangle 
				\begin{itemize}
					\item What is the Magic Triangle?
					\item Why use the Magic Triangle?
				\end{itemize}
				\item Creating new Business Model
				\begin{itemize}
					\item Initiation
					\item Ideation
					\item Integration
				\end{itemize}
			\end{itemize}
		\item ``Business Model Canvas'' \cite{bmc}
			\begin{itemize}
				\item What is the canvas used for? How was it developed? How relevant is this framework?
				\item Explain nine Building Blocks
				\begin{itemize}
					\item Value Proposition
					\item Customer Segments
					\item Channels
					\item Customer Relationships
					\item Key Activities
					\item Key Resources
					\item Key Partners
					\item Revenue Streams
					\item Cost Structures
				\end{itemize}
			\end{itemize}
	\end{itemize}
\section{Available IoT Business Model Frameworks}
	\begin{itemize}
		\item Explain changes in Business Models due to IoT Development
			\begin{itemize}
				\item From ``Business models for the Internet of Things'' \cite{dijkman}
				\begin{itemize}
					\item Explains typical types for the Business Model Canvas building blocks for IoT companies
				\end{itemize}
				\item From ``Designing Business Models for the Internet of Things'' \cite{westerlund}
				\begin{itemize}
					\item Explains how ecosystem business models can be designed (Value Design) to show ``the dynamics between the components'' or ``ow the engine works''

					\begin{enumerate}[I]
						\item \textbf{Value Drivers}
						\begin{itemize}
							\item Individual motivations
							\item Shared motivations
						\end{itemize}
						\item \textbf{Value Nodes}
						\begin{itemize}
							\item Actors
							\item Activities
							\item (Automated) processes
						\end{itemize}
						\item \textbf{Value Exchanges}
						\begin{itemize}
							\item Resources
							\item Knowledge
							\item Money
							\item Information
						\end{itemize}
						\item \textbf{Value Extracts}
						\begin{itemize}
							\item Monetization
						\end{itemize}					
					\end{enumerate}
				\end{itemize}
				\item From ``IEEE-SA Internet of Things (IoT) Ecosystem Study'' \cite{cisco}
					\begin{itemize}
						\item IoT Ecosystem
							\begin{enumerate}[I]
								\item \textbf{Where does IoT stand today}
									\begin{itemize}
										\item Players
										\item Market
										\item Technology
										\item Standardization
									\end{itemize}
								\item \textbf{Business Model View}
									\begin{itemize}
										\item New segments
										\item Open issues / Insufficiency
									\end{itemize}		
							\end{enumerate}
					\end{itemize}
				\item From ``Business Models and the Internet of Things'' \cite{fleisch}
					\begin{itemize}
						\item Based on 55 different business models patterns \cite{gassmann}
						\item IoT Business Model 1: Digitally Charged Products
							\begin{itemize}
								\item A digitally charged product ``links digital services to physical products to create a hybrid bundle that is a single whole.''
								\item \textbf{Components}
									\begin{enumerate}[a]
										\item Business Model View
										\item Physical Freemium
										\item Digital Add-on
										\item Digital Lock-in
										\item Object Self Service
										\item Remote Usage and Condition Monitoring
									\end{enumerate}
							\end{itemize}
						\item IoT Business Model 2: Sensor as a Service
							\begin{itemize}
								\item A sensor as a service business model is based on ``collecting, processing and selling for a fee the sensor data [...]''.
							\end{itemize}
					\end{itemize}
			\end{itemize}
	\end{itemize}

\section{Comparison of available IoT Business Model Frameworks}
	\begin{itemize}
		\item Text with explanations of differences between frameworks in Chapter 5
		\item If applicable: Comparison table along different attributes
		\item Introduction into case company (fictitious example)
		\item Present model of case company for each framework
	\end{itemize}

\section{Evaluations and Discussion}
	\begin{itemize} 
		\item Partly based on outcome of discussion within the seminar
		\item Critical questions regarding the relevance of the results
		\item How can the conducted comparison help IoT companies to design better business models?
	\end{itemize}

\section{Summary and Conclusions}
	\begin{itemize} 
		\item Summarize the paper content
		\item Show limitations of the paper
		\item Give indications for potential future work related to the topic of this paper
	\end{itemize}

	 \begin{thebibliography}{99}
 		 \bibitem {gassmann} O. Gassmann, K. Frankenberger, and M. Csik: \emph{Revolutionizing the business model.} in O. Gassmann and F. Schweitzer (Eds.), Management of the Fuzzy Front End of Innovation. Springer, New York, pp. 89-97, 2014, \url{http://link.springer.com/chapter/10.1007%2F978-3-319-01056-4_7}, last visit: October 06, 2016.

 		 \bibitem {bmc} Strategyzer AG: \emph{The Business Model Canvas}, \url{http://www.businessmodelgeneration.com/canvas/bmc}, last visit: October 06, 2016.

 		 \bibitem {dijkman} R.M. Dijkman, B. Sprenkels, T.Peeters, and A. Janssen: \emph{Business Models for the Internet of Things}, International Journal of Information Management, Vol 35, pp 672-678, 2015.

 		 \bibitem {fleisch} E. Fleisch, M. Winberger, F. Wortmann: \emph{Business Models and the Internet of Things}, Bosch Internet of Things & Services Lab, pp 1-19, August 2014, \url{http://www.iot-lab.ch/?page_id=10543}, last visit: October 06, 2016.

 		 \bibitem {rossi} B. Rossi: \emph{How the Internet of Things is Changing Business Models}, Information Age - Insight and analysis for IOT leaders, May 4, 2016 \url{http://www.information-age.com/it-management/strategy-and-innovation/123461371/how-internet-things-changes-business-models}, last visit: October 06, 2016.

 		 \bibitem {hui} G. Hui: \emph{How the Internet of Things Changes Business Models}, Harvard Business Review, July 24, 2014, \url{https://hbr.org/2014/07/how-the-internet-of-things-changes-business-models}, last visit: October 06, 2016.

 		 \bibitem {cisco} CISCO: \emph{IEEE-SA Internet of Things Ecosystem Study}, IEEE Standards Association, New York, 2015, \url{http://www.cisco.com/c/dam/en/us/solutions/collateral/industry-solutions/dlfe-670918525.pdf}, last visit: October 06, 2016.

 		\bibitem {westerlund} M. Westerlund, S. Leminen and M. Rajahonka: \emph{Designing Business Models for the Internet of Things} Technology Innovation Management Review, July 2014, \url{http://timreview.ca/article/807}, last visit: October 06, 2016.

 		\bibitem {ju} Jaehyeon Ju, Mi-Seon Kim and Jae-Hyeon Ahn: \emph{Prototyping Business Models for IoT Service}, Procedia Computer Science, 91, pp. 882 - 890, 2016, \url{http://www.sciencedirect.com/science/article/pii/S1877050916312911}, last visit: October 06, 2016.
	 \end{thebibliography}


% Je nach Talk Nummer wird ausschlie�lich im jeweiligen \texttt{TalkX} Verzeichnis
% gearbeitet. Bitte Dateien, Bilder etc. \textbf{nur} im Verzeichnis \texttt{TalkX}
% ablegen und bearbeiten. Die Ausarbeitung wird in die jeweilige
% \texttt{Seminar-Arbeit.tex} geschrieben. Bitte die Datei \texttt{Example.tex} als 
% Grundlage nehmen. Formatierungen, Seiteneinstellungen oder die Datei \texttt{talk.tex} 
% d�rfen nicht ge�ndert werden.

% Sollte es sich absolut nicht vermeiden lassen, da� weitere usepackages
% eingebunden werden m�ssen o.�. bitte \textbf{niemals} die Datei
% \texttt{talk.tex} bearbeiten. Daf�r vorgesehen ist die Datei
% \texttt{TalkX/MyHeader.tex}. Bitte nur im Notfall bzw. nach
% R�cksprache mit dem Betreuer von dieser M�glichkeit Gebrauch machen,
% da \LaTeX \ nicht �ber Namensr�ume verf�gt und sich somit Konflikte zwischen
% einzelnen Paketen ergeben k�nnen.

% \section{Gliederung der Arbeit}

% Die Seminar-Arbeit wird in ein Kapitel (\texttt{$\backslash$chapter}) geschrieben. 
% Zur Gliederung der Arbeit werden die Befehle
% \texttt{$\backslash$section\{\}},
% \texttt{$\backslash$subsection\{\}}, und
% \texttt{$\backslash$subsubsection\{\}}
% verwendet.

% Abs�tze m�ssen generell mit einer leeren Zeile getrennt werden und nicht mit
% \texttt{$\backslash$$\backslash$} oder \texttt{$\backslash$newline}. 
% Bitte die Befehle \texttt{$\backslash$newpage}, \texttt{$\backslash$clearpage} etc. nicht verwenden.

% Aufz�hlungen mit und ohne Nummerierung k�nnen mit den folgeneden Befehlen erstellt werden:
% \begin{quote}
%   \begin{verbatim}
%   \begin{enumerate}
%     \item ...
%     \item ...
%   \end{enumerate}

%   \begin{itemize}
%     \item ...
%     \item ...
%   \end{itemize}
%   \end{verbatim}
% \end{quote}

% F�r Beschreibungen kann der folgende Befehl benutzt werden:
% \begin{quote}
%   \begin{verbatim}
%   \begin{description}
%     \item[Begriff] Beschreibung
%     \item[Begriff] Beschreibung
%   \end{description}
%   \end{verbatim}
% \end{quote}


% \section{Bilder und Tabellen}

% Bitte \textbf{alle} Bilder ohne Dateinamenendung einbinden und das jeweilige
% Bild als \texttt{.jpg} oder \texttt{.pdf} im Verzeichnis \texttt{TalkX} ablegen. 
% Zum Einbinden kann der folgende Befehl verwendet werdet:
% \begin{quote}
%   \begin{verbatim}
%   \begin{figure}[ht]
%     \begin{center}
%     \includegraphics[scale=0.6]{TalkX/filename}
%     \end{center}
%     \caption{Beschriftung}
%     \label{label}
%   \end{figure}
%   \end{verbatim}
% \end{quote}

% \begin{figure}[ht]
% 	\begin{center}
%   \includegraphics[scale=0.6]{Example/uzh_logo_d}
%   \end{center}
%   \caption{Beschriftung}
%   \label{fig:label}
% \end{figure}

% Beim Einbinden den relativen Pfad angeben und keinen absoluten!
% Anstatt \texttt{[scale=0.6]} k�nnen auch die Parameter \texttt{[width=4cm]} oder
% \texttt{[width=0.6$\backslash$textwidth]} f�r die Skalierung der Bilder verwendet werden.
% In der schriftlichen Ausarbeitung m�ssen die Bilder mit einer Mindestaufl�sung
% von 600dpi erstellt werden.

%   \begin{table}[h]
%     \caption{Beschriftung}
%     \label{tab:label}
%   	\begin{center}
%     \begin{tabular}{|c|c|c|c|} \hline
% 	          & A & B & C \\ \hline\hline
% 	        X & 1 & 2 & 3 \\ \hline
% 	        Y & 4 & 5 & 6 \\ \hline
% 	        Z & 7 & 8 & 9 \\ \hline
% 	  \end{tabular}
% 	  \end{center}
%   \end{table}

% Die Tabelle \ref{tab:label} kann z.B. mit dem folgenden Befehl erstellt werden:
% \begin{quote}
%   \begin{verbatim}
%   \begin{table}
%     \caption{Beschriftung}
%     \label{tab:label}
%     \begin{center}
%     \begin{tabular}{|c|c|c|c|} \hline
% 	          & A & B & C \\ \hline\hline
% 	        X & 1 & 2 & 3 \\ \hline
% 	        Y & 4 & 5 & 6 \\ \hline
% 	        Z & 7 & 8 & 9 \\ \hline
%     \end{tabular}
%     \end{center}
%   \end{table}
%   \end{verbatim}
% \end{quote}

% Die Bilder und Tabellen m�ssen grunds�tzlich mit einer Beschriftung (\verb|\caption|)
% versehen werden, und im Text mit \texttt{$\backslash$ref\{label\}} referenziert werden.
% Das \texttt{caption} ist bei Bildern grunds�tzlich unten und bei Tabellen oben!


%\section{Literaturverzeichnis}

% Am Ende des Kapitels wird das Literaturverzeichnis erstellt. Bei den Bibitems
% wird \textbf{keine} Marke angegeben, es werden die automatisch erzeugten
% Marken [1],[2],... verwendet. Bei einer Referenz m�ssen grunds�tzlich die
% Autoren, der Titel, der Verlag und das Erscheinungsdatum in dem folgenden Format 
% angegeben werden:
%\begin{verbatim}
%	\bibitem {label} N. Author: Title of the document; Type of document 
%   (technical report, deliverable, Workshop/Conference Name ...), 
%   (Location, Vol. X, No. Y), Month, Year, pages, URL (if available).

%	\bibitem {label} Website title; \url{Website URL}, Month, Year of last visit.
%\end{verbatim}
% Wenn in der Referenz eine Internetadresse benutzt wird, muss diese mit dem
% Befehl \texttt{$\backslash$url\{http://...\}} angegeben werden.

% Im Text werden die Bibitems mit \texttt{$\backslash$cite\{label\}} referenziert.
% F�r alle verwendeten Arbeiten und Bilder m�ssen an der jeweiligen Stelle Referenzen
% eingef�gt werden.

% Eine ausf�hrliche Anleitung zur Referenzierung in wissenschaftlichen Arbeiten 
% findet sich im \emph{Leitfaden zur Erstellung schriftliche Seminararbeiten} \cite{leitfaden}. 

% \section{Kompilieren}

% \LaTeX \ ist bei allen g�ngigen Linux Distributionen dabei. Unter Linux wird 
% das Dokument mit \texttt{pdflatex talk.tex} im Hauptverzeichnis kompiliert, 
% was die Datei \texttt{talk.pdf} erzeugt. 

% F�r Windows ist die \TeX \ Implementation, MiKTeX (\url{http://www.miktex.org/}) zusammen mit dem \LaTeX \ Tool, TeXnicCenter (\url{http://www.toolscenter.org/}) zu empfehlen.

% Probleme, Anregungen und Fragen bzgl. der Anfertigung des Dokumentes richten
% Sie bitte per Email an die Betreuer.

% Zum Abgabetermin ist es dann nur notwendig das Verzeichnis
% \texttt{TalkX} zu packen (mit \texttt{zip} oder \texttt{tar}) und per
% Email an die Betreuer zu senden.


